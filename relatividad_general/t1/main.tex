\documentclass{article}

\usepackage[most]{tcolorbox}
\usepackage{physics}
\usepackage{graphicx}
\usepackage{float}
\usepackage{amsmath}
\usepackage{amssymb}


\usepackage[utf8]{inputenc}
\usepackage[a4paper, margin=1in]{geometry} % Controla los márgenes
\usepackage{titling}

\title{Taller \#1 Relatividad General}
\author{Manuel Garcia.}
\date{\today}

\renewcommand{\maketitlehooka}{%
  \centering
  \vspace*{0.05cm} % Espacio vertical antes del título
}

\renewcommand{\maketitlehookd}{%
  \vspace*{2cm} % Espacio vertical después de la fecha
}

\newcommand{\caja}[3]{%
  \begin{tcolorbox}[colback=#1!5!white,colframe=#1!25!black,title=#2]
    #3
  \end{tcolorbox}%
}

\begin{document}
\maketitle

\section{}

Sean dos eventos $P_1$ y $P_2$ con coordenadas espacio-temporales en un sistema inercial $\Sigma$ dadas por
\[
P_1 = (x_1^0, x_1^1, x_1^2, x_1^3), \quad P_2 = (x_2^0, x_2^1, x_2^2, x_2^3),
\]
y definamos el intervalo espacio-temporal como:
\[
\Delta S_{12}^2 = c^2(t_2 - t_1)^2 - (\vb{x}_2 - \vb{x}_1)^2.
\]
Estudiaremos tres casos distintos según el signo de $\Delta S_{12}^2$.

\textbf{a)} Intervalo tipo tiempo: $\Delta S_{12}^2 > 0$

En este caso, el intervalo es tipo tiempo, lo que significa que los dos eventos pueden estar conectados causalmente y existe un sistema de referencia $\Sigma_P$ en el cual los dos eventos ocurren en el mismo lugar del espacio. Es decir, $\vb{x}_1 = \vb{x}_2$ en dicho sistema, por lo que:

\[
\Delta S_{12}^2 = c^2 \Delta \tau^2,
\]
donde $\Delta \tau$ es el \textbf{intervalo de tiempo propio} medido en el sistema $\Sigma_P$.

Dado que $t_2 > t_1$ en el sistema $\Sigma$, entonces:
\[
\Delta t = t_2 - t_1 = \gamma(v) \Delta \tau > 0 \Rightarrow \Delta \tau > 0,
\]
lo que implica que todos los observadores inerciales relacionados por transformaciones de Lorentz (TL) también observarán que $t'_2 > t'_1$. Por tanto, \textbf{el orden temporal de los eventos es invariante relativista}.

Este caso también está asociado con el fenómeno de dilatación temporal:
\[
\Delta t = \frac{\Delta \tau}{\sqrt{1 - \frac{v^2}{c^2}}} = \gamma(v)\Delta \tau.
\]

\textbf{b)} Intervalo tipo luz: $\Delta S_{12}^2 = 0$

Aquí, el intervalo es nulo, y los eventos están conectados por una señal luminosa o por una partícula que viaja a la velocidad de la luz. Entonces:
\[
c^2(t_2 - t_1)^2 = (\vb{x}_2 - \vb{x}_1)^2 \Rightarrow \Delta S^2 = 0.
\]
En este caso, el intervalo es invariante bajo transformaciones de Lorentz. Si en el sistema $\Sigma$ se cumple que $t_2 > t_1$, entonces para cualquier otro sistema inercial $\Sigma'$ también se tendrá $t'_2 > t'_1$. Por tanto, \textbf{el orden temporal de los eventos también es invariante}.

\textbf{c)} Intervalo tipo espacio: $\Delta S_{12}^2 < 0$

Este tipo de intervalo indica que los eventos están separados más en espacio que en tiempo, es decir, no hay una conexión causal entre ellos. En este caso:
\[
\Delta S_{12}^2 < 0 \Rightarrow c^2(t_2 - t_1)^2 < (\vb{x}_2 - \vb{x}_1)^2.
\]

Esto implica que \textbf{existe un sistema de referencia} $\Sigma'$ en el cual ambos eventos son simultáneos ($t'_2 = t'_1$). Además, también existen sistemas donde el orden temporal se invierte ($t'_2 < t'_1$). Esto ocurre porque la simultaneidad no es absoluta en relatividad especial. Por tanto, \textbf{el orden temporal de eventos con intervalo tipo espacio no es un invariante relativista}.

Este hecho refleja la imposibilidad de que eventos separados por un intervalo tipo espacio tengan una relación causal, ya que requeriría una velocidad de propagación mayor a $c$.


\section{}
Sea $x(\lambda)$ la línea de universo de una partícula y $\lambda$ un parámetro.


\textbf{a) } Cuando $\lambda = \tau$, la cuadrivelocidad está definida como:
\[
U^\mu = \dv{x^\mu}{\tau}.
\]

La cuadria-aceleración es:
\[
A^\mu = \dv{U^\mu}{\tau}.
\]

Bajo la métrica de Minkowski $(+,-,-,-)$, el producto interno se define como:
\[
U \cdot A = \eta_{\mu\nu} U^\mu A^\nu = U^0 A^0 - \vb{U} \cdot \vb{A},
\]

donde $\vb{U}$ y $\vb{A}$ son las partes espaciales de los vectores.

Primero, notamos que:
\[
\dv{}{ \tau } (U^\mu U_\mu) = 0,
\]

Calculando explícitamente:
\[
\dv{}{ \tau } (U^\mu U_\mu) = 2 U_\mu \dv{U^\mu}{\tau} = 2 U_\mu A^\mu = 2 (U \cdot A).
\]
Por tanto:
\[
U \cdot A = 0.
\]

\begin{itemize}
    \item La cuadrivelocidad \( U^\mu \) se define como:
      \[
      U^\mu = \dv{x^\mu}{\tau}.
      \]

      Componentes:
      \[
      U^\mu = \left( \dv{x^0}{\tau}, \dv{x^1}{\tau}, \dv{x^2}{\tau}, \dv{x^3}{\tau} \right).
      \]

      El cuadrado de la cuadrivelocidad bajo la métrica de Minkowski \((+,-,-,-)\) es:
      \[
      U^2 = \eta_{\mu\nu} U^\mu U^\nu = (U^0)^2 - (U^1)^2 - (U^2)^2 - (U^3)^2.
      \]

      Ahora, por definición del tiempo propio:
      \[
      c^2 \dd{\tau}^2 = c^2 \dd{t}^2 - \dd{x}^2 - \dd{y}^2 - \dd{z}^2,
      \]
      lo cual implica:
      \[
      \left( \dv{x^0}{\tau} \right)^2 - \left( \dv{x^1}{\tau} \right)^2 - \left( \dv{x^2}{\tau} \right)^2 - \left( \dv{x^3}{\tau} \right)^2 = c^2,
      \]
      es decir,
      \[
      U^2 = c^2.
      \]

    \item La cuadriaceleración \( A^\mu \) se define como:
      \[
      A^\mu = \dv{U^\mu}{\tau}.
      \]

      Componentes:
      \[
      A^\mu = \left( \dv[2]{x^0}{\tau}, \dv[2]{x^1}{\tau}, \dv[2]{x^2}{\tau}, \dv[2]{x^3}{\tau} \right).
      \]

      Ahora calculemos explícitamente el producto \( U \cdot A \):
      \[
      U \cdot A = \eta_{\mu\nu} U^\mu A^\nu = U^0 A^0 - U^1 A^1 - U^2 A^2 - U^3 A^3.
      \]
      El cuadrado de la cuadriaceleración es:
\[
A^2 = \eta_{\mu\nu} A^\mu A^\nu = (A^0)^2 - (A^1)^2 - (A^2)^2 - (A^3)^2.
\]

\( A^\mu \) es ortogonal a \( U^\mu \) y \( A^\mu \) es de tipo espacio, por lo tanto:
\[
A^2 < 0.
\]
\end{itemize}






\section{}

\textbf{a) } 
En \(\Sigma\) el fotón incidente tiene cuatro-vector
\[
k^\mu \;=\;\bigl(\tfrac{\omega}{c},\,k_x,\,k_y,\,0\bigr),
\]
donde
\[
k_x = \frac{\omega}{c}\,\cos\theta_i,
\qquad
k_y = \frac{\omega}{c}\,\sin\theta_i,
\]
y \(\omega=|\vec k|\,c\).
El espejo se mueve a \(-v\,\hat x\), así que para ir a su reposo hacemos un \emph{boost} con velocidad \(+\!v\,\hat x\). Llamamos \(\beta=v/c\) y
\(\gamma=1/\sqrt{1-\beta^2}\). La transformación de Lorentz en la dirección \(x\) es
\[
\begin{cases}
k'^0 = \gamma\,(k^0 - \beta\,k_x),\\[6pt]
k'_x = \gamma\,(k_x - \beta\,k^0),\\[3pt]
k'_y = k_y\,,
\end{cases}
\]
con \(k^0=\omega/c\). Por tanto:
\[
\omega'\equiv c\,k'^0
= c\,\gamma\Bigl(\tfrac{\omega}{c}-\beta\,\tfrac{\omega}{c}\cos\theta_i\Bigr)
=\omega\,\gamma\,(1-\beta\cos\theta_i),
\]
y el coseno del ángulo en \(\Sigma'\) es
\[
\cos\theta_i'
=\frac{k'_x}{|\vec k'\,|}
=\frac{\gamma\bigl(\frac{\omega}{c}\cos\theta_i-\beta\frac{\omega}{c}\bigr)}
      {\gamma\frac{\omega}{c}(1-\beta\cos\theta_i)}
=\frac{\cos\theta_i-\beta}{1-\beta\cos\theta_i}.
\]
En el marco del espejo, la reflexión especular clásica da
\[
\theta_r'=\theta_i',
\qquad
\omega'\ \text{(frecuencia)}\ \text{invariante}.
\]
Por tanto, el vector reflejado es
\[
k_r'{}^\mu
=\Bigl(\tfrac{\omega'}{c},\,-k'_x,\,k'_y,\,0\Bigr).
\]
Ahora aplicamos la transformación inversa (boost de \(-v\hat x\)):
\[
\begin{cases}
k^0_r = \gamma\,(k'^0 + \beta\,k'_x),\\[6pt]
k_{r x} = \gamma\,(k'_x + \beta\,k'^0),\\[3pt]
k_{r y} = k'_y.
\end{cases}
\]
Calculamos
\[
k_{r x}
= \gamma\Bigl(-k'_x + \beta\,k'^0\Bigr)
= \gamma\Bigl[-\gamma\bigl(k_x-\beta k^0\bigr)
               +\beta\,\gamma\bigl(k^0-\beta k_x\bigr)\Bigr]
\]
y de forma análoga para la magnitud $|\vec k_r|$.  
Tras factorizar se llega a
\[
\boxed{
\cos\theta_r \;=\;
\frac{2\,\beta \;+\; (1+\beta^2)\,\cos\theta_i}
     {1 \;+\;2\,\beta\,\cos\theta_i\;+\;\beta^2}
}
\quad\bigl(\beta=v/c\bigr),
\]


\textbf{b)} Si \(\vec v\) es paralelo al plano \(yz\), entonces en la TL
\(\beta\) aparece sólo en las componentes \(y,z\) y \(\hat x\)
queda sin “mezclarse”. Se sigue que
\(\cos\theta_i'=\cos\theta_i\) y por simetría de la reflexión
\(\cos\theta_r'=\cos\theta_i'\). Al regresar a \(\Sigma\)
quedará \(\cos\theta_r=\cos\theta_i\), luego
\[
\theta_r=\theta_i.
\]






\section{}
\[
f^\mu \;=\;\frac{dp^\mu}{d\tau}
\;=\;\gamma(\vb u)\,\frac{dp^\mu}{dt},
\]
donde
\[
p^\mu = m_0\,U^\mu = m_0\,\gamma(\vb u)\,(c,\vb u),
\quad
\gamma(\vb u)=\frac1{\sqrt{1-\tfrac{u^2}{c^2}}}.
\]


\textbf{a) } Escribimos
\[
p^\mu = \bigl(p^0,\vb p\bigr)
         = \bigl(m_0\gamma c,\;m_0\gamma\,\vb u\bigr).
\]
Entonces
\[
\frac{dp^0}{dt}
= \frac{d}{dt}\bigl(m_0\gamma c\bigr)
= c\,\frac{d}{dt}\bigl(m_0\gamma\bigr)
= c\,\frac{dm}{dt},
\]
y
\[
\frac{d\vb p}{dt} = \vb F.
\]
Por tanto,
\[
\frac{dp^\mu}{dt}
= \bigl(c\,\tfrac{dm}{dt},\;\vb F\bigr),
\]
y
\[
f^\mu
= \gamma(\vb u)\,\frac{dp^\mu}{dt}
= \gamma(\vb u)\,
  \Bigl(c\,\frac{dm}{dt},\;\vb F\Bigr).
\]


\textbf{b) }Usamos el hecho de que la cuadrivelocidad y la cuadriaceleración satisfacen
\[
U^\mu A_\mu = 0,
\quad
U^\mu = \dv{x^\mu}{\tau}, \quad
A^\mu = \dv{U^\mu}{\tau}.
\]
Y tambien que
\[
E = p^0 c = m c^2,
\]
de modo que
\[
\frac{dE}{dt} = \frac{d}{dt}(mc^2).
\]
Por otro lado,
el trabajo por unidad de tiempo es
\(\dot W = \vb F\!\cdot\!\vb u\), y en relatividad
\(\dot W=dE/dt\). Así,
\[
\frac{d}{dt}(m c^2)
= \vb F\!\cdot\!\vb u.
\]



\textbf{c) }Si usamos la relación
\[
\frac{dE}{dt} \;=\;\vb F\!\cdot\!\vb u
\;\Longrightarrow\;
\frac{1}{c}\,\frac{dE}{dt} \;=\;\frac{1}{c}\,\vb F\!\cdot\!\vb u,
\]
y recordamos que \(E=c\,p^0\), podemos escribir
\[
\frac{dp^0}{dt}
= \frac{1}{c}\,\frac{dE}{dt}
= \frac{1}{c}\,\vb F\!\cdot\!\vb u.
\]
Entonces
\[
\frac{dp^\mu}{dt}
= \Bigl(\tfrac{1}{c}\,\vb F\!\cdot\!\vb u,\;\vb F\Bigr),
\]
y, como antes,
\[
f^\mu
= \gamma(\vb u)\,\frac{dp^\mu}{dt}
= \gamma(\vb u)\,
  \Bigl(\tfrac{1}{c}\,\vb F\!\cdot\!\vb u,\;\vb F\Bigr).
\]




\section{}

\textbf{a) }Escribimos
\[
\gamma\,u = \frac{u}{\sqrt{1-u^2/c^2}} = \Phi(t),
\]
y la ecuación es \(\dot\Phi = g\). Integrando con \(\Phi(0)=0\):
\[
\Phi(t) = g\,t,
\]
es decir
\[
\frac{u}{\sqrt{1-u^2/c^2}} = g\,t.
\]
Elevamos al cuadrado:
\[
\frac{u^2}{1-u^2/c^2} = g^2 t^2
\;\Longrightarrow\;
u^2 \;=\; \frac{g^2 t^2}{1 + \frac{g^2 t^2}{c^2}}
\;=\; \frac{(g t)^2}{1 + (g t/c)^2}\,c^2.
\]
Por tanto
\[
u(t) \;=\; \frac{g\,t}{\sqrt{1 + \bigl(\tfrac{g\,t}{c}\bigr)^2}}\,.
\]

Usamos
\(\dot x = u(t)\). Entonces
\[
x(t) = \int_0^t \frac{g\,t'}{\sqrt{1 + (g t'/c)^2}}\,dt'.
\]
Cambiamos variable \(y = (g t')/c\):
\[
x(t) = \frac{c^2}{g}
\int_0^{gt/c} \frac{y}{\sqrt{1+y^2}}\,dy
= \frac{c^2}{g}
\bigl[\sqrt{1+y^2}-1\bigr]_{0}^{gt/c}
= \frac{c^2}{g}\Bigl(\sqrt{1+(gt/c)^2}-1\Bigr).
\]
Así,
\[
\boxed{x(t) = \frac{c^2}{g}\Bigl(\sqrt{1+\bigl(\tfrac{g\,t}{c}\bigr)^2}-1\Bigr)\,.}
\]


\textbf{b)}
Para aceleración propia constante \(g\) es más sencillo usar \(\tau\). Sabemos que
\[
t(\tau) = \frac{c}{g}\sinh\Bigl(\frac{g\,\tau}{c}\Bigr),
\qquad
x(\tau) = \frac{c^2}{g}\Bigl[\cosh\Bigl(\frac{g\,\tau}{c}\Bigr)-1\Bigr].
\]
Estas expresiones satisfacen las condiciones iniciales \(t(0)=0\), \(x(0)=0\) y
\(\dot x\bigl|_{\tau=0}=0\).  

La cuadrivelocidad es
\[
U^\mu = \dv{x^\mu}{\tau}
= \Bigl(c \cosh\!\tfrac{g\tau}{c},\;
           c \sinh\!\tfrac{g\tau}{c},\;0,\;0\Bigr).
\]









\section{}
Consideremos una partícula de masa propia \(m_0\) cuya posición respecto a un sistema inercial \(\Sigma\) es \(\vb r(t)\), con velocidad \(\vb u = \dot{\vb r}\). Sobre ella actúa una fuerza central atractiva
\[
\vb F(\vb r) \;=\; -\frac{k}{r^3}\,\vb r,
\qquad r = \|\vb r\|.
\]
La fuerza cuatrivectorial es
\[
f^\mu \;=\; \dv{p^\mu}{\tau}
\;=\;\gamma(u)\,\dv{p^\mu}{t},
\]
donde
\[
p^\mu = m_0\,\gamma(u)\,(c,\;\vb u),
\qquad
\gamma(u)=\frac{1}{\sqrt{1-u^2/c^2}}.
\]
Denotamos la energía total como
\[
E \;=\; \gamma(u)\,m_0\,c^2 \;-\;\frac{k}{r}.
\]



\textbf{a) } Queremos mostrar que \(\displaystyle \frac{dE}{dt}=0\). 

1. Transformada temporal de la fuerza:
\[
\frac{dp^0}{dt}
= \frac{1}{c}\,\frac{d}{dt}\bigl(\gamma m_0 c^2\bigr)
= \frac{1}{c}\,\frac{d}{dt}\bigl(E_{\rm kin}\bigr),
\]
pero también de la parte espacial obtenemos la fuerza usual,
\(\displaystyle \frac{d\vb p}{dt}=\vb F\).

2. Relación trabajo–energía:
\[
\frac{d}{dt}\bigl(\gamma m_0 c^2\bigr)
= \vb F\cdot\vb u,
\]
puesto que \(\dot W = \vb F\cdot\vb u\) y \(\dot W = dE_{\rm kin}/dt\).

3. Cálculo de \(\vb F\cdot\vb u\):
\[
\vb F\cdot\vb u
= -\frac{k}{r^3}\,\vb r\cdot\dot{\vb r}
= -\frac{k}{r^3}\,\frac{d}{dt}\!\Bigl(\tfrac12 r^2\Bigr)
= -\frac{k}{r^2}\,\frac{dr}{dt}
= -\frac{d}{dt}\!\Bigl(\frac{k}{r}\Bigr).
\]

4. Combinando,
\[
\frac{dE}{dt}
= \frac{d}{dt}\Bigl(\gamma m_0 c^2\Bigr)
  \;-\;\frac{d}{dt}\Bigl(\frac{k}{r}\Bigr)
= \vb F\cdot\vb u \;-\;\vb F\cdot\vb u
= 0.
\]



\textbf{b) } Definimos el momento angular orbital
\[
\vb L = \vb r \times \vb p
= m_0\,\gamma(u)\,\bigl(\vb r\times\vb u\bigr).
\]
Su derivada temporal es
\[
\frac{d\vb L}{dt}
= \dot{\vb r}\times\vb p \;+\;\vb r\times\frac{d\vb p}{dt}
= \vb u\times\bigl(m_0\gamma\,\vb u\bigr)
  \;+\;\vb r\times\vb F.
\]
Pero \(\vb u\times\vb u = \vb 0\) y
\[
\vb r\times\vb F
= -\frac{k}{r^3}\,\vb r\times\vb r
= \vb 0.
\]
Así,
\[
\frac{d\vb L}{dt} = \vb 0,
\]








\setcounter{section}{8}
\section{}

\begin{enumerate}
\item[a)] Demostraciones:

\begin{itemize}
\item[1.] Antisimetría:
\begin{align*}
[A,B] &= AB - BA\,,\\
[B,A] &= BA - AB\,,\\
\Rightarrow [A,B] &= -[B,A]\,.
\end{align*}

\item[2.] Linealidad:
\begin{align*}
[A,B+C] &= A(B+C) - (B+C)A\\
&= AB + AC - BA - CA\\
&= (AB - BA) + (AC - CA)\\
&= [A,B] + [A,C]\,.
\end{align*}

\item[3.] Regla del producto:
\begin{align*}
[A,BC] &= A(BC) - (BC)A\\
&= (AB)C - B(CA)\\
&= (AB)C - B(AC)\\
&= (AB - BA)C + B(AC - CA)\\
&= [A,B]C + B[A,C]\,.
\end{align*}

\item[4.] Comportamiento bajo transposición:
\begin{align*}
[A,B]^T &= (AB - BA)^T\\
&= (AB)^T - (BA)^T\\
&= B^T A^T - A^T B^T\\
&= [B^T, A^T]\,.
\end{align*}

\item[5.] Identidad de Jacobi:
Expandimos cada término:
\begin{align*}
[A,[B,C]] &= A(BC - CB) - (BC - CB)A = ABC - ACB - BCA + CBA\,,\\
[C,[A,B]] &= C(AB - BA) - (AB - BA)C = CAB - CBA - ABC + BAC\,,\\
[B,[C,A]] &= B(CA - AC) - (CA - AC)B = BCA - BAC - CAB + ACB\,.
\end{align*}
Sumando los tres resultados:
\begin{align*}
[A,[B,C]] + [C,[A,B]] + [B,[C,A]] &= (ABC - ACB - BCA + CBA) \\
&\quad + (CAB - CBA - ABC + BAC) \\
&\quad + (BCA - BAC - CAB + ACB)\\
&= 0\,,
\end{align*}
ya que todos los términos se cancelan par a par.
\end{itemize}
\end{enumerate}


\end{document}
