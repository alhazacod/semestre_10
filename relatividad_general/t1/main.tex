\documentclass{article}

\usepackage[most]{tcolorbox}
\usepackage{physics}
\usepackage{graphicx}
\usepackage{float}
\usepackage{amsmath}
\usepackage{amssymb}


\usepackage[utf8]{inputenc}
\usepackage[a4paper, margin=1in]{geometry} % Controla los márgenes
\usepackage{titling}

\title{Taller \#1 Relatividad General}
\author{Manuel Garcia.}
\date{\today}

\renewcommand{\maketitlehooka}{%
  \centering
  \vspace*{0.05cm} % Espacio vertical antes del título
}

\renewcommand{\maketitlehookd}{%
  \vspace*{2cm} % Espacio vertical después de la fecha
}

\newcommand{\caja}[3]{%
  \begin{tcolorbox}[colback=#1!5!white,colframe=#1!25!black,title=#2]
    #3
  \end{tcolorbox}%
}

\begin{document}
\maketitle

\section{}

Sean dos eventos $P_1$ y $P_2$ con coordenadas espacio-temporales en un sistema inercial $\Sigma$ dadas por
\[
P_1 = (x_1^0, x_1^1, x_1^2, x_1^3), \quad P_2 = (x_2^0, x_2^1, x_2^2, x_2^3),
\]
y definamos el intervalo espacio-temporal como:
\[
\Delta S_{12}^2 = c^2(t_2 - t_1)^2 - (\vb{x}_2 - \vb{x}_1)^2.
\]
Estudiaremos tres casos distintos según el signo de $\Delta S_{12}^2$.

\textbf{a)} Intervalo tipo tiempo: $\Delta S_{12}^2 > 0$

En este caso, el intervalo es tipo tiempo, lo que significa que los dos eventos pueden estar conectados causalmente y existe un sistema de referencia $\Sigma_P$ en el cual los dos eventos ocurren en el mismo lugar del espacio. Es decir, $\vb{x}_1 = \vb{x}_2$ en dicho sistema, por lo que:

\[
\Delta S_{12}^2 = c^2 \Delta \tau^2,
\]
donde $\Delta \tau$ es el \textbf{intervalo de tiempo propio} medido en el sistema $\Sigma_P$.

Dado que $t_2 > t_1$ en el sistema $\Sigma$, entonces:
\[
\Delta t = t_2 - t_1 = \gamma(v) \Delta \tau > 0 \Rightarrow \Delta \tau > 0,
\]
lo que implica que todos los observadores inerciales relacionados por transformaciones de Lorentz (TL) también observarán que $t'_2 > t'_1$. Por tanto, \textbf{el orden temporal de los eventos es invariante relativista}.

Este caso también está asociado con el fenómeno de dilatación temporal:
\[
\Delta t = \frac{\Delta \tau}{\sqrt{1 - \frac{v^2}{c^2}}} = \gamma(v)\Delta \tau.
\]

\textbf{b)} Intervalo tipo luz: $\Delta S_{12}^2 = 0$

Aquí, el intervalo es nulo, y los eventos están conectados por una señal luminosa o por una partícula que viaja a la velocidad de la luz. Entonces:
\[
c^2(t_2 - t_1)^2 = (\vb{x}_2 - \vb{x}_1)^2 \Rightarrow \Delta S^2 = 0.
\]
En este caso, el intervalo es invariante bajo transformaciones de Lorentz. Si en el sistema $\Sigma$ se cumple que $t_2 > t_1$, entonces para cualquier otro sistema inercial $\Sigma'$ también se tendrá $t'_2 > t'_1$. Por tanto, \textbf{el orden temporal de los eventos también es invariante}.

\textbf{c)} Intervalo tipo espacio: $\Delta S_{12}^2 < 0$

Este tipo de intervalo indica que los eventos están separados más en espacio que en tiempo, es decir, no hay una conexión causal entre ellos. En este caso:
\[
\Delta S_{12}^2 < 0 \Rightarrow c^2(t_2 - t_1)^2 < (\vb{x}_2 - \vb{x}_1)^2.
\]

Esto implica que \textbf{existe un sistema de referencia} $\Sigma'$ en el cual ambos eventos son simultáneos ($t'_2 = t'_1$). Además, también existen sistemas donde el orden temporal se invierte ($t'_2 < t'_1$). Esto ocurre porque la simultaneidad no es absoluta en relatividad especial. Por tanto, \textbf{el orden temporal de eventos con intervalo tipo espacio no es un invariante relativista}.

Este hecho refleja la imposibilidad de que eventos separados por un intervalo tipo espacio tengan una relación causal, ya que requeriría una velocidad de propagación mayor a $c$.

\end{document}
