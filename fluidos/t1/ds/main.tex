\documentclass{article}
\usepackage{amsmath}
\usepackage{siunitx}
\title{Solución Taller 1 - Elasticidad y Fluidos}
\author{Manuel Garcia}
\date{}
\begin{document}
\maketitle

\section*{Problema 1.1: Disco compacto CD}
(a) Rapidez angular en RPM:
\[
\omega = \frac{v}{r} \times \frac{60}{2\pi} \Rightarrow
\begin{cases}
r = \SI{23}{mm} = \SI{0.023}{m}: & \omega \approx \SI{540}{rpm} \\
r = \SI{58}{mm} = \SI{0.058}{m}: & \omega \approx \SI{214}{rpm}
\end{cases}
\]

(b) Revoluciones en 74 min 33 s:
\[
N = \frac{v \cdot T}{2\pi \cdot r_{\text{avg}}} \approx \frac{1.3 \cdot 4473}{2\pi \cdot 0.0405} \approx \boxed{22800 \text{ revoluciones}}
\]

(c) Aceleración angular promedio:
\[
\alpha = \frac{\omega_2 - \omega_1}{T} \approx \frac{22.41 - 56.52}{4473} \approx \boxed{\SI{-7.63e-3}{rad/s^2}}
\]

\section*{Problema 1.2: Molécula de oxígeno}
(A) Momento de inercia:
\[
I = \frac{1}{2} m d^2 = \frac{1}{2} (2.66 \times 10^{-26})(1.21 \times 10^{-10})^2 \approx \boxed{\SI{1.95e-46}{kg\cdot m^2}}
\]

(B) Energía cinética rotacional:
\[
K = \frac{1}{2} I \omega^2 = \frac{1}{2}(1.95 \times 10^{-46})(4.60 \times 10^{12})^2 \approx \boxed{\SI{2.06e-21}{J}}
\]

\section*{Problema 1.3: Estrella de neutrones}
Conservación de momento angular:
\[
\frac{T_2}{T_1} = \left(\frac{R_2}{R_1}\right)^2 \Rightarrow T_2 = 30 \times \left(\frac{3}{10000}\right)^2 \approx \boxed{\SI{0.234}{s}}
\]

\section*{Problema 1.5: Cilindro bajo deformación}
1. Deformación axial:
\[
\epsilon_z = \frac{F}{A E} = \frac{-2000 \times 10^3}{\pi (0.1)^2 \cdot 200 \times 10^9} \approx \boxed{-3.18 \times 10^{-4}}, \quad \Delta L \approx \boxed{\SI{-0.318}{mm}}
\]

2. Deformación volumétrica:
\[
\frac{\Delta V}{V_0} = -\frac{P}{K} = -\frac{50 \times 10^6}{166.67 \times 10^9} \approx \boxed{-3 \times 10^{-4}}
\]

3. Contracción radial ($\nu = 0.3$):
\[
\Delta r = r_0 \left(\nu \epsilon_z - \frac{P(1-2\nu)}{E}\right) \approx \boxed{\SI{-0.445}{\micro m}}
\]

\section*{Problema 1.6: Fosa de las Marianas}
1. Cambio de volumen:
\[
\Delta V = -\frac{P}{K} V_0 = -\frac{1.13 \times 10^8}{2.2 \times 10^9} \approx \boxed{\SI{-0.0514}{m^3}}
\]

2. Densidad:
\[
\rho = \frac{\rho_0}{1 + \Delta V/V} \approx \frac{1030}{0.9486} \approx \boxed{\SI{1086}{kg/m^3}}
\]

3. El agua puede considerarse incompresible para pequeños cambios de presión, pero no a grandes profundidades.

\section*{Problema 1.7: Alambre estirado}
1. Constante elástica:
\[
k = \frac{Y A}{L}
\]

2. Trabajo realizado:
\[
W = \frac{1}{2} k (\Delta L)^2 = \frac{Y A (\Delta L)^2}{2 L}
\]

\section*{Problema 1.8: Huesos de buzo}
(a) Presión requerida:
\[
\Delta P = K \cdot \frac{\Delta V}{V} = 15 \times 10^9 \cdot 0.001 \approx \boxed{\SI{148}{atm}}
\]

(b) Profundidad:
\[
h = \frac{\Delta P}{\rho g} = \frac{15 \times 10^6}{1 \times 10^4} = \boxed{\SI{1500}{m}}
\]

\section*{Problema 1.9: Varilla vertical}
1. Área transversal:
\[
A = \frac{P g}{\sigma_{\text{max}} - \rho L g} \approx \boxed{\SI{140.2}{cm^2}}
\]

2. Alargamiento total:
\[
\Delta L = \frac{P g L}{E A} + \frac{\rho g L^2}{2 E} \approx \boxed{\SI{8.18}{cm}}
\]

\section*{Problema 1.10: Tendón de Aquiles}
2. Fuerza (asumiendo T = 2×peso):
\[
T = 2 \times 75 \times 9.81 \approx \boxed{\SI{1471}{N}} \quad (2.0 \times \text{peso})
\]

3. Estiramiento:
\[
\Delta L = \frac{T L}{A Y} = \frac{1471 \cdot 0.25}{78 \times 10^{-6} \cdot 1470 \times 10^6} \approx \boxed{\SI{3.2}{mm}}
\]

\end{document}
